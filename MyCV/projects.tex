\section{\ifnativelang Projets libres\else Open-source projects\fi}

\ifnativelang
\hspace{3cm} Certaines de mes contributions sont accessibles sur \textcolor{color1}{\textbf{\href{github.com/JBeaudaux}{github}}}. Plus d'informations via les hyperliens.
\else
\hspace{3cm} Some of my contributions are available on Github \textcolor{color1}{\textbf{\href{https://github.com/JBeaudaux/Blueprint}{\ExternalLink}}}. See more infos using the hyperlink.
\fi

\vspace{\ItemsSpacing}

\ifnativelang
\cvitem{\textbf{Open Heart}}{
Coach d'activit\'e physique pour r\'ehabilitation cardiaque, via application mobile et bracelet connect\'e.
\textcolor{color1}{\href{https://github.com/OpenHeartHHC}{\ExternalLink}}}
{\hspace{3.1cm}\textbf{Prix du "meilleur objet connect\'e" - Hackathon Hacking Health 2016}}
{}{}
\else
\cvitem{\textbf{Open Heart}}{
\textbf{Physical activity coach for at-risk patients :} Conception of a mobile application and a connected bracelet for an optimized and controlled practice of cardiac rehabilitation.
\textcolor{color1}{\href{https://github.com/OpenHeartHHC}{\ExternalLink}}}
{\hspace{3.1cm}\textbf{"Best connected object solution" prize - Hacking Health Camp 2016}}
{}{}
\fi

\vspace{\ItemsSpacing}

\ifnativelang
\cvitem{\textbf{Blue-print}}{
Montre connect\'ee pour patients, dot\'ee d'un bouton d'alarme et d'un m\'ecanisme de d\'etection de chutes.
\textcolor{color1}{\href{https://github.com/JBeaudaux/Blueprint}{\ExternalLink}}}{}
{}{}
\else
\cvitem{\textbf{Blue-prints}}{
\textbf{Life-logging solution for activity anomaly detection :} Development of a connected watch for at-risk patients, equipped with an alarm button and a fall-detection mechanism. 
\textcolor{color1}{\href{https://github.com/JBeaudaux/Blueprint}{\ExternalLink}}}{}
{}{}
\fi

\vspace{\ItemsSpacing}

\ifnativelang
\cvitem{\textbf{IoTLab}}{
{Outil de suivi des performances et d'un d\'emonstrateur pour plateforme d'exp\'erimentation IoT.
%\textbf{Plateforme d'exp\'erimentation pour l'Internet des objets :} D\'evelopement d'un outil de monitorage et de cartographie des performances (RTT, pertes, conso.). Conception d'un d\'emonstrateur pour l'IoT.
\textcolor{color1}{\href{www.iot-lab.info}{\ExternalLink}}}{}}{}{}
\else
\cvitem{\textbf{IoTLab}}{
{\textbf{Internet of things experimental platform :} Developement of a tool to monitor and map performances (RTT, loss-rate, nodes energy consumption, etc.). Conception of a demonstrator for the IoT.
\textcolor{color1}{\href{www.iot-lab.info}{\ExternalLink}}}{}}{}{}
\fi
